\chapter{Resumen}

En los últimos años la inclusión de las nuevas tecnologías en el senderismo y otros deportes de montaña ha hecho que estos sufran una profesionalización. En el senderismo los usuarios se ponen metas más peligrosas conforme aumenta su experiencia en la montaña, lo que hace que se incremente la posibilidad de sufrir accidentes. Además, los guías de senderismo no disponen de herramientas suficientes en el mercado actual para realizar un buen análisis de la situación del grupo de expedición.

El presente \ac{TFG} surge como una solución a esta problemática y tiene la intención de cubrir el vacío que existe en el mercado actual de herramientas para realizar un análisis grupal en expediciones en la montaña. En este proyecto se plantea la construcción de un sistema formado por un dispositivo físico, una aplicación móvil y una plataforma \textit{web} que permita monitorizar la situación del grupo de expedición con el objetivo principal de maximizar la seguridad del mismo. El dispositivo físico está equipado con una serie de sensores que captan información acerca del entorno del usuario y la envían a la aplicación móvil para que realice un procesamiento de dicha información. La plataforma \textit{web} proporciona una vista detallada de las expediciones realizadas y en curso.

En este proyecto, los usuarios de la expedición podrán visualizar tanto en tiempo real como de forma histórica información relevante acerca de la expedición que están realizando. Los guías además, tendrán una vista global de la expedición (visualizando el estado de todos los participantes de la expedición) y de las alertas que en esta se generan con el objetivo de actuar lo más rápidamente posible para evitar situaciones de riesgo.

\chapter{Abstract}

In the last few years the inclusion of new technologies in hiking and other mountain sports has made them suffer a professionalization. In hiking, users set more dangerous goals as their experience in mountaineering increases, which increases the likelihood of accidents. In addition, in today's market there are no tools for hiking guides to perform a good analysis of the situation of the expedition group.

This \ac{TFG} comes as a solution to this problem and is aimed at filling the gap in tools for group analysis in mountain expeditions that exists in today's market. This project involves the construction of a system consisting of a physical device, a mobile application and a web platform to monitor the situation of the expedition group with the main objective of maximizing its safety. The physical device is equipped with a number of sensors which capture information about the user's environment and send it to the mobile application for processing this information. The web platform provides a detailed view of finished and ongoing expeditions.

In this project, the users of the expedition will be able to visualize in real time as well as in a historical way relevant information about the expedition they are currently carrying out. The guides will also have a global view of the expedition (visualizing the status of all participants in the expedition) and of the alerts that are generated in the expedition in order to act as quickly as possible to avoid risk situations.

\chapter{Agradecimientos}

En primer lugar a mis padres Rafa y Oliva, por darme la oportunidad de labrar mi futuro en la profesión de mis sueños y confiar en mí y en mis posibilidades. A mis hermanos Diego y Óscar, por su sonrisa en los momentos más difíciles.

A Lidia, por comenzar y finalizar la etapa universitaria juntos y por ser mi gran apoyo a lo largo de la misma.

A mis compañeros, por hacer las clases más amenas y por darme un buen motivo para asistir a ellas.

Por último y no menos importante, a todo el profesorado de la Escuela Superior de Informática, por transmitirme aún más su pasión por la informática. En especial a mi director de proyecto, Javier Alonso Albusac Jiménez, por su implicación a lo largo de este trabajo, su disponibilidad total para la resolución de dudas tanto presencialmente como por videoconferencia y por su disposición para revisar el proyecto los fines de semana y festivos.

\begin{flushright}
Iván.
\end{flushright}